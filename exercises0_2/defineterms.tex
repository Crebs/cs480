\documentclass{article}
\usepackage{times,amssymb,amsmath}
%% $Header: /usr/local/cvsroot/courses/cs480/latex/cs480preamble.tex,v 1.1 2006/10/04 18:23:43 neffr Exp $

%% this file should be included with a command like
%% %% $Header: /usr/local/cvsroot/courses/cs480/latex/cs480preamble.tex,v 1.1 2006/10/04 18:23:43 neffr Exp $

%% this file should be included with a command like
%% %% $Header: /usr/local/cvsroot/courses/cs480/latex/cs480preamble.tex,v 1.1 2006/10/04 18:23:43 neffr Exp $

%% this file should be included with a command like
%% \input{/home/cs480/latex/cs480preamble.tex}

%
% Set lengths
%
\setlength{\oddsidemargin}{.25in}
\setlength{\evensidemargin}{.25in}
\setlength{\textwidth}{6in}
\setlength{\topmargin}{-0.4in}
\setlength{\textheight}{8.5in}

%
% mathify-- ensure argument is in math mode
%
\newcommand{\mathify}[1]{\ifmmode{#1}\else\mbox{$#1$}\fi}

%
% Header box to go at the top of the first page
%
\def\subjnum{CS480}
\def\subjname{Computational Theory}

\newcommand{\headerbox}[3]{
  \renewcommand{\thepage}{\arabic{page}}
  \noindent
  \begin{center}
    \framebox{
      \vbox{
        \hbox to 5.78in { {\bf \subjnum} \hfill {\bf \subjname} }
        \vspace{4mm}
        \hbox to 5.78in { {\Large \hfill #1  \hfill} }
        \vspace{2mm}
        \hbox to 5.78in { {\it #2 \hfill #3} }
        }
      }
  \end{center}
  \vspace*{4mm}
  }

\newcommand{\exploration}[2]{\headerbox{#1}{#2}{Exploration}}
\newcommand{\assignment}[3]{\headerbox{#1}{Assignment #1}{#2}{#3}}
\newcommand{\exercises}[2]{\headerbox{#1}{#2}{Exemplary Answers}}
\newcommand{\inclassexercise}[2]{\headerbox{#1}{In-class exercise}{#2}}
\newcommand{\handout}[3]{\headerbox{#1}{Handout #2}{#3}}
\newcommand{\handin}[3]{\headerbox{#1}{#2}{#3}}

%
% Useful symbols
%
\newcommand{\qed}{\rule{7pt}{7pt}}
\newcommand{\ihat}{\hat{\imath}}
\newcommand{\jhat}{\hat{\jmath}}
\newcommand{\Nat}{\bf N}                        % natural numbers
\newcommand{\Int}{\mathbf{Z}}                   % integers
\newcommand{\Bool}{\it Bool}                   % booleans
\newcommand{\true}{\tt t}
\newcommand{\false}{\tt f}
\newcommand{\I}{\cal I}                         % interpretations
\newcommand{\M}{\cal M}                         % meaning functions
\newcommand{\A}{\cal A}                         % arithmetic interpretation
\newcommand{\B}{\cal B}                         % binary word interpretation
\newcommand{\TIME}{\mathop{\rm TIME}\nolimits}
\newcommand{\NTIME}{\mathop{\rm NTIME}\nolimits}
\newcommand{\SPACE}{\mathop{\rm SPACE}\nolimits}
\newcommand{\NSPACE}{\mathop{\rm NSPACE}\nolimits}
\newcommand{\union}{\cup}
\newcommand{\intersect}{\cap}
% \newcommand{\implies}{\Rightarrow}

% 
% Useful functions
%
\newcommand{\abs}[1]{\mathify{\left| #1 \right|}}
\renewcommand{\Pr}[1]{\mathify{\mbox{Pr}\left[#1\right]}}
\newcommand{\Exp}[1]{\mathify{\mbox{Exp}\left[#1\right]}}
\newcommand{\set}[1]{\mathify{\left\{ #1 \right\}}}
\newcommand{\cset}[2]{\set{#1\ :\ #2}}  % a conditional notation to define sets
\newcommand{\lset}[2]{\set{#1,\ldots,#2}} % set {from,...,to}
\newcommand{\suchthat}{\vert}
\newcommand{\st}{\suchthat}

%
% For pseudo-code
%
\newcommand{\FOR}{{\bf for}}
\newcommand{\TO}{{\bf to}}
\newcommand{\DO}{{\bf do}}
\newcommand{\WHILE}{{\bf while}}
\newcommand{\AND}{{\bf and}}
\newcommand{\IF}{{\bf if}}
\newcommand{\THEN}{{\bf then}}
\newcommand{\ELSE}{{\bf else}}

%
% Useful environments-- theorem-like
%
\newtheorem{theorem}{Theorem}
\newtheorem{corollary}[theorem]{Corollary}
\newtheorem{lemma}[theorem]{Lemma}
\newtheorem{observation}[theorem]{Observation}
\newtheorem{proposition}[theorem]{Proposition}
\newtheorem{definition}[theorem]{Definition}
\newtheorem{claim}[theorem]{Claim}
\newtheorem{fact}[theorem]{Fact}
\newtheorem{assumption}[theorem]{Assumption}

%
% Useful environments for proofs
%
\newenvironment{proof}{\noindent{\bf Proof:}\hspace*{1em}}{\qed\bigskip}
\newenvironment{proof-sketch}{\noindent{\bf Sketch of Proof:}\hspace*{1em}}{\qed\bigskip}
\newenvironment{proof-idea}{\noindent{\bf Proof Idea:} \hspace*{1em}}{\qed\bigskip}
\newenvironment{proof-of-lemma}[1]{\noindent{\bf Proof of Lemma #1:}\hspace*{1em}}{\qed\bigskip}
\newenvironment{proof-attempt}{\noindent{\bf Proof Attempt:}\hspace*{1em}}{\qed\bigskip}
\newenvironment{proofof}[1]{\noindent{\bf Proof}
of #1:\hspace*{1em}}{\qed\bigskip}
%\newenvironment{remark}{\noindent{\bf Remark:}\hspace*{1em}}{\bigskip}

%
% abbreviations
%
\newcommand{\ie}{{\em i.e.}}
\newcommand{\etc}{{\em etc.}}
\newcommand{\eg}{{\em e.g.}}
%\newcommand{\wlog}{\em w.l.o.g.}
\newcommand{\cf}{{\em cf.}}
\newcommand{\viz}{{\em viz.}}

\newcommand{\eqdef}{\stackrel{\rm def}{=}}      % ``equals by definition''
\newcommand{\hint}{{\em Hint}:\ }              % for in-line hints
\newcommand{\note}{{\em Note}:\ }              % for in-line notes
\newcommand{\remark}{{\em Remark}\/:\ }        % for in-line remarks  

%
% CS480/theocomp specific symbols
%
\newcommand{\bigO}O
\newcommand{\emptystring}{\epsilon}              % empty string
\def\P{{\cal P}}
\def\NP{{\cal NP}}
\def\PP{{\cal BPP}}
\def\IP{{\cal IP}}
\def\RP{{\cal RP}}
\newcommand{\bit}{\set{0,1}}
\newcommand{\strings}{\bit^*}

%
% problems and subproblems
%
\newcounter{exercise}
\newcounter{problem}
%\newcounter{subproblem}[problem]
\newcounter{subproblem}
\renewcommand{\theexercise}{\arabic{exercise}}
\renewcommand{\theproblem}{\arabic{problem}}
\renewcommand{\thesubproblem}{\arabic{subproblem}}
\newenvironment{exercise} {\stepcounter{exercise} \textbf{Exercise
    \theexercise}:} {\vspace{.1in}}
\newenvironment{problem} {\stepcounter{problem} \textbf{Problem
    \theproblem}:} {\vspace{.1in}}
\newenvironment{subproblem} {\stepcounter{subproblem} 
    \thesubproblem :} {}
\newenvironment{solution} {\textbf{Solution
    \theproblem}:} {\vspace{.1in}}

% 
% machines
%
\newcommand{\rep}[1]{\left\langle #1 \right\rangle}


%%% Local Variables: 
%%% mode: latex
%%% TeX-master: t
%%% End: 


%
% Set lengths
%
\setlength{\oddsidemargin}{.25in}
\setlength{\evensidemargin}{.25in}
\setlength{\textwidth}{6in}
\setlength{\topmargin}{-0.4in}
\setlength{\textheight}{8.5in}

%
% mathify-- ensure argument is in math mode
%
\newcommand{\mathify}[1]{\ifmmode{#1}\else\mbox{$#1$}\fi}

%
% Header box to go at the top of the first page
%
\def\subjnum{CS480}
\def\subjname{Computational Theory}

\newcommand{\headerbox}[3]{
  \renewcommand{\thepage}{\arabic{page}}
  \noindent
  \begin{center}
    \framebox{
      \vbox{
        \hbox to 5.78in { {\bf \subjnum} \hfill {\bf \subjname} }
        \vspace{4mm}
        \hbox to 5.78in { {\Large \hfill #1  \hfill} }
        \vspace{2mm}
        \hbox to 5.78in { {\it #2 \hfill #3} }
        }
      }
  \end{center}
  \vspace*{4mm}
  }

\newcommand{\exploration}[2]{\headerbox{#1}{#2}{Exploration}}
\newcommand{\assignment}[3]{\headerbox{#1}{Assignment #1}{#2}{#3}}
\newcommand{\exercises}[2]{\headerbox{#1}{#2}{Exemplary Answers}}
\newcommand{\inclassexercise}[2]{\headerbox{#1}{In-class exercise}{#2}}
\newcommand{\handout}[3]{\headerbox{#1}{Handout #2}{#3}}
\newcommand{\handin}[3]{\headerbox{#1}{#2}{#3}}

%
% Useful symbols
%
\newcommand{\qed}{\rule{7pt}{7pt}}
\newcommand{\ihat}{\hat{\imath}}
\newcommand{\jhat}{\hat{\jmath}}
\newcommand{\Nat}{\bf N}                        % natural numbers
\newcommand{\Int}{\mathbf{Z}}                   % integers
\newcommand{\Bool}{\it Bool}                   % booleans
\newcommand{\true}{\tt t}
\newcommand{\false}{\tt f}
\newcommand{\I}{\cal I}                         % interpretations
\newcommand{\M}{\cal M}                         % meaning functions
\newcommand{\A}{\cal A}                         % arithmetic interpretation
\newcommand{\B}{\cal B}                         % binary word interpretation
\newcommand{\TIME}{\mathop{\rm TIME}\nolimits}
\newcommand{\NTIME}{\mathop{\rm NTIME}\nolimits}
\newcommand{\SPACE}{\mathop{\rm SPACE}\nolimits}
\newcommand{\NSPACE}{\mathop{\rm NSPACE}\nolimits}
\newcommand{\union}{\cup}
\newcommand{\intersect}{\cap}
% \newcommand{\implies}{\Rightarrow}

% 
% Useful functions
%
\newcommand{\abs}[1]{\mathify{\left| #1 \right|}}
\renewcommand{\Pr}[1]{\mathify{\mbox{Pr}\left[#1\right]}}
\newcommand{\Exp}[1]{\mathify{\mbox{Exp}\left[#1\right]}}
\newcommand{\set}[1]{\mathify{\left\{ #1 \right\}}}
\newcommand{\cset}[2]{\set{#1\ :\ #2}}  % a conditional notation to define sets
\newcommand{\lset}[2]{\set{#1,\ldots,#2}} % set {from,...,to}
\newcommand{\suchthat}{\vert}
\newcommand{\st}{\suchthat}

%
% For pseudo-code
%
\newcommand{\FOR}{{\bf for}}
\newcommand{\TO}{{\bf to}}
\newcommand{\DO}{{\bf do}}
\newcommand{\WHILE}{{\bf while}}
\newcommand{\AND}{{\bf and}}
\newcommand{\IF}{{\bf if}}
\newcommand{\THEN}{{\bf then}}
\newcommand{\ELSE}{{\bf else}}

%
% Useful environments-- theorem-like
%
\newtheorem{theorem}{Theorem}
\newtheorem{corollary}[theorem]{Corollary}
\newtheorem{lemma}[theorem]{Lemma}
\newtheorem{observation}[theorem]{Observation}
\newtheorem{proposition}[theorem]{Proposition}
\newtheorem{definition}[theorem]{Definition}
\newtheorem{claim}[theorem]{Claim}
\newtheorem{fact}[theorem]{Fact}
\newtheorem{assumption}[theorem]{Assumption}

%
% Useful environments for proofs
%
\newenvironment{proof}{\noindent{\bf Proof:}\hspace*{1em}}{\qed\bigskip}
\newenvironment{proof-sketch}{\noindent{\bf Sketch of Proof:}\hspace*{1em}}{\qed\bigskip}
\newenvironment{proof-idea}{\noindent{\bf Proof Idea:} \hspace*{1em}}{\qed\bigskip}
\newenvironment{proof-of-lemma}[1]{\noindent{\bf Proof of Lemma #1:}\hspace*{1em}}{\qed\bigskip}
\newenvironment{proof-attempt}{\noindent{\bf Proof Attempt:}\hspace*{1em}}{\qed\bigskip}
\newenvironment{proofof}[1]{\noindent{\bf Proof}
of #1:\hspace*{1em}}{\qed\bigskip}
%\newenvironment{remark}{\noindent{\bf Remark:}\hspace*{1em}}{\bigskip}

%
% abbreviations
%
\newcommand{\ie}{{\em i.e.}}
\newcommand{\etc}{{\em etc.}}
\newcommand{\eg}{{\em e.g.}}
%\newcommand{\wlog}{\em w.l.o.g.}
\newcommand{\cf}{{\em cf.}}
\newcommand{\viz}{{\em viz.}}

\newcommand{\eqdef}{\stackrel{\rm def}{=}}      % ``equals by definition''
\newcommand{\hint}{{\em Hint}:\ }              % for in-line hints
\newcommand{\note}{{\em Note}:\ }              % for in-line notes
\newcommand{\remark}{{\em Remark}\/:\ }        % for in-line remarks  

%
% CS480/theocomp specific symbols
%
\newcommand{\bigO}O
\newcommand{\emptystring}{\epsilon}              % empty string
\def\P{{\cal P}}
\def\NP{{\cal NP}}
\def\PP{{\cal BPP}}
\def\IP{{\cal IP}}
\def\RP{{\cal RP}}
\newcommand{\bit}{\set{0,1}}
\newcommand{\strings}{\bit^*}

%
% problems and subproblems
%
\newcounter{exercise}
\newcounter{problem}
%\newcounter{subproblem}[problem]
\newcounter{subproblem}
\renewcommand{\theexercise}{\arabic{exercise}}
\renewcommand{\theproblem}{\arabic{problem}}
\renewcommand{\thesubproblem}{\arabic{subproblem}}
\newenvironment{exercise} {\stepcounter{exercise} \textbf{Exercise
    \theexercise}:} {\vspace{.1in}}
\newenvironment{problem} {\stepcounter{problem} \textbf{Problem
    \theproblem}:} {\vspace{.1in}}
\newenvironment{subproblem} {\stepcounter{subproblem} 
    \thesubproblem :} {}
\newenvironment{solution} {\textbf{Solution
    \theproblem}:} {\vspace{.1in}}

% 
% machines
%
\newcommand{\rep}[1]{\left\langle #1 \right\rangle}


%%% Local Variables: 
%%% mode: latex
%%% TeX-master: t
%%% End: 


%
% Set lengths
%
\setlength{\oddsidemargin}{.25in}
\setlength{\evensidemargin}{.25in}
\setlength{\textwidth}{6in}
\setlength{\topmargin}{-0.4in}
\setlength{\textheight}{8.5in}

%
% mathify-- ensure argument is in math mode
%
\newcommand{\mathify}[1]{\ifmmode{#1}\else\mbox{$#1$}\fi}

%
% Header box to go at the top of the first page
%
\def\subjnum{CS480}
\def\subjname{Computational Theory}

\newcommand{\headerbox}[3]{
  \renewcommand{\thepage}{\arabic{page}}
  \noindent
  \begin{center}
    \framebox{
      \vbox{
        \hbox to 5.78in { {\bf \subjnum} \hfill {\bf \subjname} }
        \vspace{4mm}
        \hbox to 5.78in { {\Large \hfill #1  \hfill} }
        \vspace{2mm}
        \hbox to 5.78in { {\it #2 \hfill #3} }
        }
      }
  \end{center}
  \vspace*{4mm}
  }

\newcommand{\exploration}[2]{\headerbox{#1}{#2}{Exploration}}
\newcommand{\assignment}[3]{\headerbox{#1}{Assignment #1}{#2}{#3}}
\newcommand{\exercises}[2]{\headerbox{#1}{#2}{Exemplary Answers}}
\newcommand{\inclassexercise}[2]{\headerbox{#1}{In-class exercise}{#2}}
\newcommand{\handout}[3]{\headerbox{#1}{Handout #2}{#3}}
\newcommand{\handin}[3]{\headerbox{#1}{#2}{#3}}

%
% Useful symbols
%
\newcommand{\qed}{\rule{7pt}{7pt}}
\newcommand{\ihat}{\hat{\imath}}
\newcommand{\jhat}{\hat{\jmath}}
\newcommand{\Nat}{\bf N}                        % natural numbers
\newcommand{\Int}{\mathbf{Z}}                   % integers
\newcommand{\Bool}{\it Bool}                   % booleans
\newcommand{\true}{\tt t}
\newcommand{\false}{\tt f}
\newcommand{\I}{\cal I}                         % interpretations
\newcommand{\M}{\cal M}                         % meaning functions
\newcommand{\A}{\cal A}                         % arithmetic interpretation
\newcommand{\B}{\cal B}                         % binary word interpretation
\newcommand{\TIME}{\mathop{\rm TIME}\nolimits}
\newcommand{\NTIME}{\mathop{\rm NTIME}\nolimits}
\newcommand{\SPACE}{\mathop{\rm SPACE}\nolimits}
\newcommand{\NSPACE}{\mathop{\rm NSPACE}\nolimits}
\newcommand{\union}{\cup}
\newcommand{\intersect}{\cap}
% \newcommand{\implies}{\Rightarrow}

% 
% Useful functions
%
\newcommand{\abs}[1]{\mathify{\left| #1 \right|}}
\renewcommand{\Pr}[1]{\mathify{\mbox{Pr}\left[#1\right]}}
\newcommand{\Exp}[1]{\mathify{\mbox{Exp}\left[#1\right]}}
\newcommand{\set}[1]{\mathify{\left\{ #1 \right\}}}
\newcommand{\cset}[2]{\set{#1\ :\ #2}}  % a conditional notation to define sets
\newcommand{\lset}[2]{\set{#1,\ldots,#2}} % set {from,...,to}
\newcommand{\suchthat}{\vert}
\newcommand{\st}{\suchthat}

%
% For pseudo-code
%
\newcommand{\FOR}{{\bf for}}
\newcommand{\TO}{{\bf to}}
\newcommand{\DO}{{\bf do}}
\newcommand{\WHILE}{{\bf while}}
\newcommand{\AND}{{\bf and}}
\newcommand{\IF}{{\bf if}}
\newcommand{\THEN}{{\bf then}}
\newcommand{\ELSE}{{\bf else}}

%
% Useful environments-- theorem-like
%
\newtheorem{theorem}{Theorem}
\newtheorem{corollary}[theorem]{Corollary}
\newtheorem{lemma}[theorem]{Lemma}
\newtheorem{observation}[theorem]{Observation}
\newtheorem{proposition}[theorem]{Proposition}
\newtheorem{definition}[theorem]{Definition}
\newtheorem{claim}[theorem]{Claim}
\newtheorem{fact}[theorem]{Fact}
\newtheorem{assumption}[theorem]{Assumption}

%
% Useful environments for proofs
%
\newenvironment{proof}{\noindent{\bf Proof:}\hspace*{1em}}{\qed\bigskip}
\newenvironment{proof-sketch}{\noindent{\bf Sketch of Proof:}\hspace*{1em}}{\qed\bigskip}
\newenvironment{proof-idea}{\noindent{\bf Proof Idea:} \hspace*{1em}}{\qed\bigskip}
\newenvironment{proof-of-lemma}[1]{\noindent{\bf Proof of Lemma #1:}\hspace*{1em}}{\qed\bigskip}
\newenvironment{proof-attempt}{\noindent{\bf Proof Attempt:}\hspace*{1em}}{\qed\bigskip}
\newenvironment{proofof}[1]{\noindent{\bf Proof}
of #1:\hspace*{1em}}{\qed\bigskip}
%\newenvironment{remark}{\noindent{\bf Remark:}\hspace*{1em}}{\bigskip}

%
% abbreviations
%
\newcommand{\ie}{{\em i.e.}}
\newcommand{\etc}{{\em etc.}}
\newcommand{\eg}{{\em e.g.}}
%\newcommand{\wlog}{\em w.l.o.g.}
\newcommand{\cf}{{\em cf.}}
\newcommand{\viz}{{\em viz.}}

\newcommand{\eqdef}{\stackrel{\rm def}{=}}      % ``equals by definition''
\newcommand{\hint}{{\em Hint}:\ }              % for in-line hints
\newcommand{\note}{{\em Note}:\ }              % for in-line notes
\newcommand{\remark}{{\em Remark}\/:\ }        % for in-line remarks  

%
% CS480/theocomp specific symbols
%
\newcommand{\bigO}O
\newcommand{\emptystring}{\epsilon}              % empty string
\def\P{{\cal P}}
\def\NP{{\cal NP}}
\def\PP{{\cal BPP}}
\def\IP{{\cal IP}}
\def\RP{{\cal RP}}
\newcommand{\bit}{\set{0,1}}
\newcommand{\strings}{\bit^*}

%
% problems and subproblems
%
\newcounter{exercise}
\newcounter{problem}
%\newcounter{subproblem}[problem]
\newcounter{subproblem}
\renewcommand{\theexercise}{\arabic{exercise}}
\renewcommand{\theproblem}{\arabic{problem}}
\renewcommand{\thesubproblem}{\arabic{subproblem}}
\newenvironment{exercise} {\stepcounter{exercise} \textbf{Exercise
    \theexercise}:} {\vspace{.1in}}
\newenvironment{problem} {\stepcounter{problem} \textbf{Problem
    \theproblem}:} {\vspace{.1in}}
\newenvironment{subproblem} {\stepcounter{subproblem} 
    \thesubproblem :} {}
\newenvironment{solution} {\textbf{Solution
    \theproblem}:} {\vspace{.1in}}

% 
% machines
%
\newcommand{\rep}[1]{\left\langle #1 \right\rangle}


%%% Local Variables: 
%%% mode: latex
%%% TeX-master: t
%%% End: 


%%%
%%% To produce a Device Independent (.dvi) file from this one:
%%%
%%%    latex defineterms.tex
%%%
%%% To view the resulting .dvi file:
%%%
%%%    xdvi defineterms.dvi
%%%
%%% Alternatively, to convert the DVI file to PDF:
%%%
%%%    dvipdf defineterms.dvi
%%%

%%% a useful command, particularly if we change our minds later...
\newcommand{\term}[1]{\textsc{#1}}

\begin{document}

\headerbox{0.2 Mathematical Notions and Terminology}

\large

Define each of these terms:

\bigskip \indent
\begin{itemize}
\item set
  \subitem An unordered collection of unique objects, called \term{element}s.  Sets are usually designated by their \term{element}s listed within curly braces, such as $A=\set{a,b,c}$.
\item element(s)
  \subitem Any particular object contained in a \term{set} or \term{sequence}.
  Also known as \term{member}.
\item member(s)
  \subitem Any particular object contained in a \term{set}.
  Also known as \term{element}.
\item subset
  \subitem A \term{set} which does not include any \term{element}s that are not included in another \term{set}, known as the \term{superset}.  Thus, if \term{A} is a subset of \term{B}, then there are no elements in \term{A} that are not found in \term{B}.
\item proper subset
  \subitem A \term{subset} which does not include all the \term{element}s of its \term{superset}.  That is, a \term{proper subset} will have at least one \term{element} less than its \term{superset}.
\item multiset
  \subitem An unordered collection of \term{element}s which may or may not be repeated.
\item infinite set
  \subitem A \term{set} which contains an infinite number of \term{element}s.
\item natural numbers $\mathbb{N}$
  \subitem Generally refers to the positive \term{integer}s, $\set{1,2,3,\ldots}$, though in some contexts it is the non-negative \term{integer}s, $\set{0,1,2, \ldots}$.
\item integers $\mathbb{Z}$
  \subitem The positive and negative whole numbers and zero, $\set{\ldots,-2,-1,0,1,2,\ldots}$.  This is equivalent to the \term{natural number}s, their opposites ({\ie} negatives), and zero.
\item empty set $\varnothing$
  \subitem The \term{set} which has no \term{element}s.  Often marked $\varnothing$, but sometimes denoted $\set{}$.
\item union $\cup$
  \subitem A \term{set operation} which generates a new \term{set} containing all the \term{element}s of each source \term{set}.
\item intersection $\cap$
  \subitem A \term{set operation} which generates a new \term{set} containing only the \term{element}s which are found in each of the source \term{set}s.
\item complement $\overline{A}$
  \subitem A \term{set operation}, which generates a new \term{set} containing all the \term{element}s in the universe of discourse which are \emph{not} found in the source \term{set}.  Frequently denoted by a line over the source set, thus the complement of $A$ is $\overline{A}$.
\item compliment
  \subitem An expression of esteem, respect, affection, or admiration; especially : an admiring remark. A thing that should be given to Brother Neff frequently. 
\item Venn diagram
  \subitem Shows all hypothetically possible logical relations between a finite collection of \term{sets} using a circle as a representation of a single \term{set}
\item sequence
  \subitem A \term{function} from a \term{subset} of the \term{set} of integers (usually either the \term{set} \{0, 1, 2 ...\} or the \term{set} \{1, 2, 3, ... \}) to a \term{set} S. 
\item tuple(s)
   \subitem Finite \term{sequence(s)}
\item k-tuple
   \subitem A \term{sequence} of $k$ \term{elements}, where $k$ is a positive integer
\item pair
   \subitem A 2-\term{tuple}
\item power set 
  \subitem The \term{power set} of the \term{set} $A$ is the \term{set} of all \term{subsets} of $A$; denoted by the notation $\mathcal{P}\left(A\right)$

\item Cartesian product $A \times B$
  \subitem The \term{set} of all ordered pairs $(a, b)$, where $a \in A$ and $b \in B$. Thus, $A \times B = \{(a,b) | a\in A \text{and} \ b\in B\}$. It is important to point out that $A \times B$ is generally not the same as $B \times A$.

\item cross product $A \times B$
  \subitem The \term{cross product} $a \times b$ is defined as a vector $c$ that is perpendicular to both $a$ and $b$, with a direction given by the right-hand rule and a magnitude equal to the area of the parallelogram that the vectors span. The cross product is defined by the formula: $a \times b = a b \sin \theta n$

\item function
  \subitem A \term{function} $f$ from $A$ to $B$ is an assignment of exactly one element of $B$ to each element of $A$. We write $f(a) = b$ if $b$ is the unique element of $B$ assigned by the function $f$ to the element $a$ of $A$. If $f$ is a function from $A$ to $B$, we write $f \colon A \to B$. Also known as \term{mapping}.

\item mapping
  \subitem A \term{mapping} $f$ from $A$ to $B$ is an assignment of exactly one element of $B$ to each element of $A$. Also known as \term{function}.

\item domain
   \subitem The \term{set} of possible inputs to a \term{function}
\item codomain
   \subitem The \term{set} of possible outputs to a \term{function}
\item range
   \subitem The \term{set} of all images from a \term{subset} of a \term{function's} \term{domain}, where an image is the \term{set} of all outputs obtained when the \term{function} is evaluated at each \term{element} of the \term{subset}
\item onto
   \subitem A \term{function} that uses all the \term{elements} of the \term{codomain}
\item surjective
   \subitem \term{onto}
\item surjection
   \subitem What a \term{function} is called when it is \term{onto}
\item one-to-one
   \subitem A \term{function} is \term{one-to-one} iff $f(a) = f(b)$ implies that $a = b$ for all $a$ and $b$. It never maps two different \term{elements} to the same place.
\item injective
   \subitem \term{one-to-one}
\item injection
   \subitem What a \term{function} is called when it is \term{one-to-one}
\end{itemize}

\bigskip \indent
\begin{itemize}
\item 1-to-1 correspondence
   \subitem A \term{function} that is both \term{one-to-one} and \term{onto}
\item bijective
 
\item bijection

\item arguments

\item k-ary function

\item arity

\item unary function

\item binary function

\item infix notation

\item prefix notation

\item postfix notation

\item predicate/property

\end{itemize}

\bigskip \indent
\begin{itemize}
\item relation $R$

\item k-ary relation

\item k-ary relation on A

\item binary relation

\item equivalence relation

\item reflexive

\item symmetric

\item transitive

\item undirected graph

\end{itemize}

\bigskip \indent
\begin{itemize}
\item graph

\item nodes

\item vertices

\item edges

\item degree

\item labeled graph

\item subgraph

\item path

\item simple path

\item connected

\item cycle

\item simple cycle

\item tree

\item root

\item leaves

\item directed graph

\item outdegree

\item indegree

\item directed path

\item strongly connected

\end{itemize}
   
\bigskip \indent
\begin{itemize}
\item \emph{alphabet} $\Sigma$

\item symbols

\item string over an alphabet

\item length $\abs{w}$

\item empty string $\varepsilon$

\item reverse $w^\mathcal{R}$

\item substring

\item concatenation

\item lexicographic ordering

\item language

\end{itemize}

\bigskip \indent
\begin{itemize}
\item Boolean logic

\item Boolean values

\item Boolean operations

\item negation/NOT $\neg$

\item conjunction/AND $\wedge$

\item disjunction/OR $\vee$

\item exclusive or/XOR $\oplus$

\item equality (operation) $\leftrightarrow$

\item implication (operation) $\rightarrow$

\item operands

\item distributive law for AND and OR

\end{itemize}

\end{document}

%%% Local Variables: 
%%% mode: latex
%%% TeX-master: t
%%% compile-command: "latex defineterms.tex && dvipdf defineterms.dvi"
%%% End: 
