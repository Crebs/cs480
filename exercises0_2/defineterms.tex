\documentclass{article}
\usepackage{times,amssymb,amsmath}
%% $Header: /usr/local/cvsroot/courses/cs480/latex/cs480preamble.tex,v 1.1 2006/10/04 18:23:43 neffr Exp $

%% this file should be included with a command like
%% %% $Header: /usr/local/cvsroot/courses/cs480/latex/cs480preamble.tex,v 1.1 2006/10/04 18:23:43 neffr Exp $

%% this file should be included with a command like
%% %% $Header: /usr/local/cvsroot/courses/cs480/latex/cs480preamble.tex,v 1.1 2006/10/04 18:23:43 neffr Exp $

%% this file should be included with a command like
%% \input{/home/cs480/latex/cs480preamble.tex}

%
% Set lengths
%
\setlength{\oddsidemargin}{.25in}
\setlength{\evensidemargin}{.25in}
\setlength{\textwidth}{6in}
\setlength{\topmargin}{-0.4in}
\setlength{\textheight}{8.5in}

%
% mathify-- ensure argument is in math mode
%
\newcommand{\mathify}[1]{\ifmmode{#1}\else\mbox{$#1$}\fi}

%
% Header box to go at the top of the first page
%
\def\subjnum{CS480}
\def\subjname{Computational Theory}

\newcommand{\headerbox}[3]{
  \renewcommand{\thepage}{\arabic{page}}
  \noindent
  \begin{center}
    \framebox{
      \vbox{
        \hbox to 5.78in { {\bf \subjnum} \hfill {\bf \subjname} }
        \vspace{4mm}
        \hbox to 5.78in { {\Large \hfill #1  \hfill} }
        \vspace{2mm}
        \hbox to 5.78in { {\it #2 \hfill #3} }
        }
      }
  \end{center}
  \vspace*{4mm}
  }

\newcommand{\exploration}[2]{\headerbox{#1}{#2}{Exploration}}
\newcommand{\assignment}[3]{\headerbox{#1}{Assignment #1}{#2}{#3}}
\newcommand{\exercises}[2]{\headerbox{#1}{#2}{Exemplary Answers}}
\newcommand{\inclassexercise}[2]{\headerbox{#1}{In-class exercise}{#2}}
\newcommand{\handout}[3]{\headerbox{#1}{Handout #2}{#3}}
\newcommand{\handin}[3]{\headerbox{#1}{#2}{#3}}

%
% Useful symbols
%
\newcommand{\qed}{\rule{7pt}{7pt}}
\newcommand{\ihat}{\hat{\imath}}
\newcommand{\jhat}{\hat{\jmath}}
\newcommand{\Nat}{\bf N}                        % natural numbers
\newcommand{\Int}{\mathbf{Z}}                   % integers
\newcommand{\Bool}{\it Bool}                   % booleans
\newcommand{\true}{\tt t}
\newcommand{\false}{\tt f}
\newcommand{\I}{\cal I}                         % interpretations
\newcommand{\M}{\cal M}                         % meaning functions
\newcommand{\A}{\cal A}                         % arithmetic interpretation
\newcommand{\B}{\cal B}                         % binary word interpretation
\newcommand{\TIME}{\mathop{\rm TIME}\nolimits}
\newcommand{\NTIME}{\mathop{\rm NTIME}\nolimits}
\newcommand{\SPACE}{\mathop{\rm SPACE}\nolimits}
\newcommand{\NSPACE}{\mathop{\rm NSPACE}\nolimits}
\newcommand{\union}{\cup}
\newcommand{\intersect}{\cap}
% \newcommand{\implies}{\Rightarrow}

% 
% Useful functions
%
\newcommand{\abs}[1]{\mathify{\left| #1 \right|}}
\renewcommand{\Pr}[1]{\mathify{\mbox{Pr}\left[#1\right]}}
\newcommand{\Exp}[1]{\mathify{\mbox{Exp}\left[#1\right]}}
\newcommand{\set}[1]{\mathify{\left\{ #1 \right\}}}
\newcommand{\cset}[2]{\set{#1\ :\ #2}}  % a conditional notation to define sets
\newcommand{\lset}[2]{\set{#1,\ldots,#2}} % set {from,...,to}
\newcommand{\suchthat}{\vert}
\newcommand{\st}{\suchthat}

%
% For pseudo-code
%
\newcommand{\FOR}{{\bf for}}
\newcommand{\TO}{{\bf to}}
\newcommand{\DO}{{\bf do}}
\newcommand{\WHILE}{{\bf while}}
\newcommand{\AND}{{\bf and}}
\newcommand{\IF}{{\bf if}}
\newcommand{\THEN}{{\bf then}}
\newcommand{\ELSE}{{\bf else}}

%
% Useful environments-- theorem-like
%
\newtheorem{theorem}{Theorem}
\newtheorem{corollary}[theorem]{Corollary}
\newtheorem{lemma}[theorem]{Lemma}
\newtheorem{observation}[theorem]{Observation}
\newtheorem{proposition}[theorem]{Proposition}
\newtheorem{definition}[theorem]{Definition}
\newtheorem{claim}[theorem]{Claim}
\newtheorem{fact}[theorem]{Fact}
\newtheorem{assumption}[theorem]{Assumption}

%
% Useful environments for proofs
%
\newenvironment{proof}{\noindent{\bf Proof:}\hspace*{1em}}{\qed\bigskip}
\newenvironment{proof-sketch}{\noindent{\bf Sketch of Proof:}\hspace*{1em}}{\qed\bigskip}
\newenvironment{proof-idea}{\noindent{\bf Proof Idea:} \hspace*{1em}}{\qed\bigskip}
\newenvironment{proof-of-lemma}[1]{\noindent{\bf Proof of Lemma #1:}\hspace*{1em}}{\qed\bigskip}
\newenvironment{proof-attempt}{\noindent{\bf Proof Attempt:}\hspace*{1em}}{\qed\bigskip}
\newenvironment{proofof}[1]{\noindent{\bf Proof}
of #1:\hspace*{1em}}{\qed\bigskip}
%\newenvironment{remark}{\noindent{\bf Remark:}\hspace*{1em}}{\bigskip}

%
% abbreviations
%
\newcommand{\ie}{{\em i.e.}}
\newcommand{\etc}{{\em etc.}}
\newcommand{\eg}{{\em e.g.}}
%\newcommand{\wlog}{\em w.l.o.g.}
\newcommand{\cf}{{\em cf.}}
\newcommand{\viz}{{\em viz.}}

\newcommand{\eqdef}{\stackrel{\rm def}{=}}      % ``equals by definition''
\newcommand{\hint}{{\em Hint}:\ }              % for in-line hints
\newcommand{\note}{{\em Note}:\ }              % for in-line notes
\newcommand{\remark}{{\em Remark}\/:\ }        % for in-line remarks  

%
% CS480/theocomp specific symbols
%
\newcommand{\bigO}O
\newcommand{\emptystring}{\epsilon}              % empty string
\def\P{{\cal P}}
\def\NP{{\cal NP}}
\def\PP{{\cal BPP}}
\def\IP{{\cal IP}}
\def\RP{{\cal RP}}
\newcommand{\bit}{\set{0,1}}
\newcommand{\strings}{\bit^*}

%
% problems and subproblems
%
\newcounter{exercise}
\newcounter{problem}
%\newcounter{subproblem}[problem]
\newcounter{subproblem}
\renewcommand{\theexercise}{\arabic{exercise}}
\renewcommand{\theproblem}{\arabic{problem}}
\renewcommand{\thesubproblem}{\arabic{subproblem}}
\newenvironment{exercise} {\stepcounter{exercise} \textbf{Exercise
    \theexercise}:} {\vspace{.1in}}
\newenvironment{problem} {\stepcounter{problem} \textbf{Problem
    \theproblem}:} {\vspace{.1in}}
\newenvironment{subproblem} {\stepcounter{subproblem} 
    \thesubproblem :} {}
\newenvironment{solution} {\textbf{Solution
    \theproblem}:} {\vspace{.1in}}

% 
% machines
%
\newcommand{\rep}[1]{\left\langle #1 \right\rangle}


%%% Local Variables: 
%%% mode: latex
%%% TeX-master: t
%%% End: 


%
% Set lengths
%
\setlength{\oddsidemargin}{.25in}
\setlength{\evensidemargin}{.25in}
\setlength{\textwidth}{6in}
\setlength{\topmargin}{-0.4in}
\setlength{\textheight}{8.5in}

%
% mathify-- ensure argument is in math mode
%
\newcommand{\mathify}[1]{\ifmmode{#1}\else\mbox{$#1$}\fi}

%
% Header box to go at the top of the first page
%
\def\subjnum{CS480}
\def\subjname{Computational Theory}

\newcommand{\headerbox}[3]{
  \renewcommand{\thepage}{\arabic{page}}
  \noindent
  \begin{center}
    \framebox{
      \vbox{
        \hbox to 5.78in { {\bf \subjnum} \hfill {\bf \subjname} }
        \vspace{4mm}
        \hbox to 5.78in { {\Large \hfill #1  \hfill} }
        \vspace{2mm}
        \hbox to 5.78in { {\it #2 \hfill #3} }
        }
      }
  \end{center}
  \vspace*{4mm}
  }

\newcommand{\exploration}[2]{\headerbox{#1}{#2}{Exploration}}
\newcommand{\assignment}[3]{\headerbox{#1}{Assignment #1}{#2}{#3}}
\newcommand{\exercises}[2]{\headerbox{#1}{#2}{Exemplary Answers}}
\newcommand{\inclassexercise}[2]{\headerbox{#1}{In-class exercise}{#2}}
\newcommand{\handout}[3]{\headerbox{#1}{Handout #2}{#3}}
\newcommand{\handin}[3]{\headerbox{#1}{#2}{#3}}

%
% Useful symbols
%
\newcommand{\qed}{\rule{7pt}{7pt}}
\newcommand{\ihat}{\hat{\imath}}
\newcommand{\jhat}{\hat{\jmath}}
\newcommand{\Nat}{\bf N}                        % natural numbers
\newcommand{\Int}{\mathbf{Z}}                   % integers
\newcommand{\Bool}{\it Bool}                   % booleans
\newcommand{\true}{\tt t}
\newcommand{\false}{\tt f}
\newcommand{\I}{\cal I}                         % interpretations
\newcommand{\M}{\cal M}                         % meaning functions
\newcommand{\A}{\cal A}                         % arithmetic interpretation
\newcommand{\B}{\cal B}                         % binary word interpretation
\newcommand{\TIME}{\mathop{\rm TIME}\nolimits}
\newcommand{\NTIME}{\mathop{\rm NTIME}\nolimits}
\newcommand{\SPACE}{\mathop{\rm SPACE}\nolimits}
\newcommand{\NSPACE}{\mathop{\rm NSPACE}\nolimits}
\newcommand{\union}{\cup}
\newcommand{\intersect}{\cap}
% \newcommand{\implies}{\Rightarrow}

% 
% Useful functions
%
\newcommand{\abs}[1]{\mathify{\left| #1 \right|}}
\renewcommand{\Pr}[1]{\mathify{\mbox{Pr}\left[#1\right]}}
\newcommand{\Exp}[1]{\mathify{\mbox{Exp}\left[#1\right]}}
\newcommand{\set}[1]{\mathify{\left\{ #1 \right\}}}
\newcommand{\cset}[2]{\set{#1\ :\ #2}}  % a conditional notation to define sets
\newcommand{\lset}[2]{\set{#1,\ldots,#2}} % set {from,...,to}
\newcommand{\suchthat}{\vert}
\newcommand{\st}{\suchthat}

%
% For pseudo-code
%
\newcommand{\FOR}{{\bf for}}
\newcommand{\TO}{{\bf to}}
\newcommand{\DO}{{\bf do}}
\newcommand{\WHILE}{{\bf while}}
\newcommand{\AND}{{\bf and}}
\newcommand{\IF}{{\bf if}}
\newcommand{\THEN}{{\bf then}}
\newcommand{\ELSE}{{\bf else}}

%
% Useful environments-- theorem-like
%
\newtheorem{theorem}{Theorem}
\newtheorem{corollary}[theorem]{Corollary}
\newtheorem{lemma}[theorem]{Lemma}
\newtheorem{observation}[theorem]{Observation}
\newtheorem{proposition}[theorem]{Proposition}
\newtheorem{definition}[theorem]{Definition}
\newtheorem{claim}[theorem]{Claim}
\newtheorem{fact}[theorem]{Fact}
\newtheorem{assumption}[theorem]{Assumption}

%
% Useful environments for proofs
%
\newenvironment{proof}{\noindent{\bf Proof:}\hspace*{1em}}{\qed\bigskip}
\newenvironment{proof-sketch}{\noindent{\bf Sketch of Proof:}\hspace*{1em}}{\qed\bigskip}
\newenvironment{proof-idea}{\noindent{\bf Proof Idea:} \hspace*{1em}}{\qed\bigskip}
\newenvironment{proof-of-lemma}[1]{\noindent{\bf Proof of Lemma #1:}\hspace*{1em}}{\qed\bigskip}
\newenvironment{proof-attempt}{\noindent{\bf Proof Attempt:}\hspace*{1em}}{\qed\bigskip}
\newenvironment{proofof}[1]{\noindent{\bf Proof}
of #1:\hspace*{1em}}{\qed\bigskip}
%\newenvironment{remark}{\noindent{\bf Remark:}\hspace*{1em}}{\bigskip}

%
% abbreviations
%
\newcommand{\ie}{{\em i.e.}}
\newcommand{\etc}{{\em etc.}}
\newcommand{\eg}{{\em e.g.}}
%\newcommand{\wlog}{\em w.l.o.g.}
\newcommand{\cf}{{\em cf.}}
\newcommand{\viz}{{\em viz.}}

\newcommand{\eqdef}{\stackrel{\rm def}{=}}      % ``equals by definition''
\newcommand{\hint}{{\em Hint}:\ }              % for in-line hints
\newcommand{\note}{{\em Note}:\ }              % for in-line notes
\newcommand{\remark}{{\em Remark}\/:\ }        % for in-line remarks  

%
% CS480/theocomp specific symbols
%
\newcommand{\bigO}O
\newcommand{\emptystring}{\epsilon}              % empty string
\def\P{{\cal P}}
\def\NP{{\cal NP}}
\def\PP{{\cal BPP}}
\def\IP{{\cal IP}}
\def\RP{{\cal RP}}
\newcommand{\bit}{\set{0,1}}
\newcommand{\strings}{\bit^*}

%
% problems and subproblems
%
\newcounter{exercise}
\newcounter{problem}
%\newcounter{subproblem}[problem]
\newcounter{subproblem}
\renewcommand{\theexercise}{\arabic{exercise}}
\renewcommand{\theproblem}{\arabic{problem}}
\renewcommand{\thesubproblem}{\arabic{subproblem}}
\newenvironment{exercise} {\stepcounter{exercise} \textbf{Exercise
    \theexercise}:} {\vspace{.1in}}
\newenvironment{problem} {\stepcounter{problem} \textbf{Problem
    \theproblem}:} {\vspace{.1in}}
\newenvironment{subproblem} {\stepcounter{subproblem} 
    \thesubproblem :} {}
\newenvironment{solution} {\textbf{Solution
    \theproblem}:} {\vspace{.1in}}

% 
% machines
%
\newcommand{\rep}[1]{\left\langle #1 \right\rangle}


%%% Local Variables: 
%%% mode: latex
%%% TeX-master: t
%%% End: 


%
% Set lengths
%
\setlength{\oddsidemargin}{.25in}
\setlength{\evensidemargin}{.25in}
\setlength{\textwidth}{6in}
\setlength{\topmargin}{-0.4in}
\setlength{\textheight}{8.5in}

%
% mathify-- ensure argument is in math mode
%
\newcommand{\mathify}[1]{\ifmmode{#1}\else\mbox{$#1$}\fi}

%
% Header box to go at the top of the first page
%
\def\subjnum{CS480}
\def\subjname{Computational Theory}

\newcommand{\headerbox}[3]{
  \renewcommand{\thepage}{\arabic{page}}
  \noindent
  \begin{center}
    \framebox{
      \vbox{
        \hbox to 5.78in { {\bf \subjnum} \hfill {\bf \subjname} }
        \vspace{4mm}
        \hbox to 5.78in { {\Large \hfill #1  \hfill} }
        \vspace{2mm}
        \hbox to 5.78in { {\it #2 \hfill #3} }
        }
      }
  \end{center}
  \vspace*{4mm}
  }

\newcommand{\exploration}[2]{\headerbox{#1}{#2}{Exploration}}
\newcommand{\assignment}[3]{\headerbox{#1}{Assignment #1}{#2}{#3}}
\newcommand{\exercises}[2]{\headerbox{#1}{#2}{Exemplary Answers}}
\newcommand{\inclassexercise}[2]{\headerbox{#1}{In-class exercise}{#2}}
\newcommand{\handout}[3]{\headerbox{#1}{Handout #2}{#3}}
\newcommand{\handin}[3]{\headerbox{#1}{#2}{#3}}

%
% Useful symbols
%
\newcommand{\qed}{\rule{7pt}{7pt}}
\newcommand{\ihat}{\hat{\imath}}
\newcommand{\jhat}{\hat{\jmath}}
\newcommand{\Nat}{\bf N}                        % natural numbers
\newcommand{\Int}{\mathbf{Z}}                   % integers
\newcommand{\Bool}{\it Bool}                   % booleans
\newcommand{\true}{\tt t}
\newcommand{\false}{\tt f}
\newcommand{\I}{\cal I}                         % interpretations
\newcommand{\M}{\cal M}                         % meaning functions
\newcommand{\A}{\cal A}                         % arithmetic interpretation
\newcommand{\B}{\cal B}                         % binary word interpretation
\newcommand{\TIME}{\mathop{\rm TIME}\nolimits}
\newcommand{\NTIME}{\mathop{\rm NTIME}\nolimits}
\newcommand{\SPACE}{\mathop{\rm SPACE}\nolimits}
\newcommand{\NSPACE}{\mathop{\rm NSPACE}\nolimits}
\newcommand{\union}{\cup}
\newcommand{\intersect}{\cap}
% \newcommand{\implies}{\Rightarrow}

% 
% Useful functions
%
\newcommand{\abs}[1]{\mathify{\left| #1 \right|}}
\renewcommand{\Pr}[1]{\mathify{\mbox{Pr}\left[#1\right]}}
\newcommand{\Exp}[1]{\mathify{\mbox{Exp}\left[#1\right]}}
\newcommand{\set}[1]{\mathify{\left\{ #1 \right\}}}
\newcommand{\cset}[2]{\set{#1\ :\ #2}}  % a conditional notation to define sets
\newcommand{\lset}[2]{\set{#1,\ldots,#2}} % set {from,...,to}
\newcommand{\suchthat}{\vert}
\newcommand{\st}{\suchthat}

%
% For pseudo-code
%
\newcommand{\FOR}{{\bf for}}
\newcommand{\TO}{{\bf to}}
\newcommand{\DO}{{\bf do}}
\newcommand{\WHILE}{{\bf while}}
\newcommand{\AND}{{\bf and}}
\newcommand{\IF}{{\bf if}}
\newcommand{\THEN}{{\bf then}}
\newcommand{\ELSE}{{\bf else}}

%
% Useful environments-- theorem-like
%
\newtheorem{theorem}{Theorem}
\newtheorem{corollary}[theorem]{Corollary}
\newtheorem{lemma}[theorem]{Lemma}
\newtheorem{observation}[theorem]{Observation}
\newtheorem{proposition}[theorem]{Proposition}
\newtheorem{definition}[theorem]{Definition}
\newtheorem{claim}[theorem]{Claim}
\newtheorem{fact}[theorem]{Fact}
\newtheorem{assumption}[theorem]{Assumption}

%
% Useful environments for proofs
%
\newenvironment{proof}{\noindent{\bf Proof:}\hspace*{1em}}{\qed\bigskip}
\newenvironment{proof-sketch}{\noindent{\bf Sketch of Proof:}\hspace*{1em}}{\qed\bigskip}
\newenvironment{proof-idea}{\noindent{\bf Proof Idea:} \hspace*{1em}}{\qed\bigskip}
\newenvironment{proof-of-lemma}[1]{\noindent{\bf Proof of Lemma #1:}\hspace*{1em}}{\qed\bigskip}
\newenvironment{proof-attempt}{\noindent{\bf Proof Attempt:}\hspace*{1em}}{\qed\bigskip}
\newenvironment{proofof}[1]{\noindent{\bf Proof}
of #1:\hspace*{1em}}{\qed\bigskip}
%\newenvironment{remark}{\noindent{\bf Remark:}\hspace*{1em}}{\bigskip}

%
% abbreviations
%
\newcommand{\ie}{{\em i.e.}}
\newcommand{\etc}{{\em etc.}}
\newcommand{\eg}{{\em e.g.}}
%\newcommand{\wlog}{\em w.l.o.g.}
\newcommand{\cf}{{\em cf.}}
\newcommand{\viz}{{\em viz.}}

\newcommand{\eqdef}{\stackrel{\rm def}{=}}      % ``equals by definition''
\newcommand{\hint}{{\em Hint}:\ }              % for in-line hints
\newcommand{\note}{{\em Note}:\ }              % for in-line notes
\newcommand{\remark}{{\em Remark}\/:\ }        % for in-line remarks  

%
% CS480/theocomp specific symbols
%
\newcommand{\bigO}O
\newcommand{\emptystring}{\epsilon}              % empty string
\def\P{{\cal P}}
\def\NP{{\cal NP}}
\def\PP{{\cal BPP}}
\def\IP{{\cal IP}}
\def\RP{{\cal RP}}
\newcommand{\bit}{\set{0,1}}
\newcommand{\strings}{\bit^*}

%
% problems and subproblems
%
\newcounter{exercise}
\newcounter{problem}
%\newcounter{subproblem}[problem]
\newcounter{subproblem}
\renewcommand{\theexercise}{\arabic{exercise}}
\renewcommand{\theproblem}{\arabic{problem}}
\renewcommand{\thesubproblem}{\arabic{subproblem}}
\newenvironment{exercise} {\stepcounter{exercise} \textbf{Exercise
    \theexercise}:} {\vspace{.1in}}
\newenvironment{problem} {\stepcounter{problem} \textbf{Problem
    \theproblem}:} {\vspace{.1in}}
\newenvironment{subproblem} {\stepcounter{subproblem} 
    \thesubproblem :} {}
\newenvironment{solution} {\textbf{Solution
    \theproblem}:} {\vspace{.1in}}

% 
% machines
%
\newcommand{\rep}[1]{\left\langle #1 \right\rangle}


%%% Local Variables: 
%%% mode: latex
%%% TeX-master: t
%%% End: 


%%%
%%% To produce a Device Independent (.dvi) file from this one:
%%%
%%%    latex defineterms.tex
%%%
%%% To view the resulting .dvi file:
%%%
%%%    xdvi defineterms.dvi
%%%
%%% Alternatively, to convert the DVI file to PDF:
%%%
%%%    dvipdf defineterms.dvi
%%%

%%% a useful command, particularly if we change our minds later...
\newcommand{\term}[1]{\textsc{#1}}

\begin{document}

\headerbox{0.2 Mathematical Notions and Terminology}

\large

Define each of these terms:

\bigskip \indent
\begin{itemize}
\item set
  \subitem An unordered collection of unique objects, called \term{element}s.  Sets are usually designated by their \term{element}s listed within curly braces, such as $A=\set{a,b,c}$.
\item element(s)
  \subitem Any particular object contained in a \term{set} or \term{sequence}.
  Also known as \term{member}.
\item member(s)
  \subitem Any particular object contained in a \term{set}.
  Also known as \term{element}.
\item subset
  \subitem A \term{set} which does not include any \term{element}s that are not included in another \term{set}, known as the \term{superset}.  Thus, if \term{A} is a subset of \term{B}, then there are no elements in \term{A} that are not found in \term{B}.
\item proper subset
  \subitem A \term{subset} which does not include all the \term{element}s of its \term{superset}.  That is, a \term{proper subset} will have at least one \term{element} less than its \term{superset}.
\item multiset
  \subitem An unordered collection of \term{element}s which may or may not be repeated.
\item infinite set
  \subitem A \term{set} which contains an infinite number of \term{element}s.
\item natural numbers $\mathbb{N}$
  \subitem Generally refers to the positive \term{integer}s, $\set{1,2,3,\ldots}$, though in some contexts it is the non-negative \term{integer}s, $\set{0,1,2, \ldots}$.
\item integers $\mathbb{Z}$
  \subitem The positive and negative whole numbers and zero, $\set{\ldots,-2,-1,0,1,2,\ldots}$.  This is equivalent to the \term{natural number}s, their opposites ({\ie} negatives), and zero.
\item empty set $\varnothing$
  \subitem The \term{set} which has no \term{element}s.  Often marked $\varnothing$, but sometimes denoted $\set{}$.
\item union $\cup$
  \subitem A \term{set operation} which generates a new \term{set} containing all the \term{element}s of each source \term{set}.
\item intersection $\cap$
  \subitem A \term{set operation} which generates a new \term{set} containing only the \term{element}s which are found in each of the source \term{set}s.
\item complement $\overline{A}$
  \subitem A \term{set operation}, which generates a new \term{set} containing all the \term{element}s in the universe of discourse which are \emph{not} found in the source \term{set}.  Frequently denoted by a line over the source set, thus the complement of $A$ is $\overline{A}$.
\item compliment
  \subitem A polite expression of praise or admiration. A thing that should be given to Brother Neff frequently. 
\item Venn diagram
  \subitem Shows all hypothetically possible logical relations between a finite collection of \term{sets} using a circle as a representation of a single \term{set}
\item sequence
  \subitem A \term{function} from a \term{subset} of the \term{set} of integers (usually either the \term{set} \{0, 1, 2 ...\} or the \term{set} \{1, 2, 3, ... \}) to a \term{set} S. 
\item tuple(s)
   \subitem Finite \term{sequence(s)}
\item k-tuple
   \subitem A \term{sequence} of $k$ \term{elements}, where $k$ is a positive integer
\item pair
   \subitem A \term{sequence} with 2 elements or a 2-\term{tuple}
\item power set 
  \subitem The \term{power set} of the \term{set} $A$ is the \term{set} of all \term{subsets} of $A$; denoted by the notation $\mathcal{P}\left(A\right)$
\item Cartesian product $A \times B$
  \subitem The \term{set} of all ordered \term{pair}s $(a, b)$, where $a \in A$ and $b \in B$. Thus, $A \times B = \{(a,b) | a\in A \text{and} \ b\in B\}$. It is important to point out that $A \times B$ is generally not the same as $B \times A$.
\item cross product $A \times B$
  \subitem The \term{set} of all ordered \term{pair}s $(a, b)$, where $a \in A$ and $b \in B$. Also known as a \term{Cartesian product}.
  %%The \term{cross product} $a \times b$ is defined as a vector $c$ that is perpendicular to both $a$ and $b$, with a direction given by the right-hand rule and a magnitude equal to the area of the parallelogram that the vectors span. The cross product is defined by the formula: $a \times b = a b \sin \theta n$
\item function
  \subitem A \term{function} $f$ from $A$ to $B$ is an assignment of exactly one \term{element} of $B$ to each \term{element} of $A$. We write $f(a) = b$ if $b$ is the unique \term{element} of $B$ assigned by the \term{function} $f$ to the \term{element} $a$ of $A$. If $f$ is a \term{function} from $A$ to $B$, we write $f \colon A \to B$. Also known as a \term{mapping}.
\item mapping
  \subitem A \term{mapping} $f$ from $A$ to $B$ is an assignment of exactly one \term{element} of $B$ to each \term{element} of $A$. Also known as a \term{function}.
\item domain
   \subitem The \term{set} of possible inputs to a \term{function}
\item codomain
   \subitem The \term{set} of possible outputs to a \term{function}
\item range
   \subitem The \term{set} of all images from a \term{subset} of a \term{function's} \term{domain}, where an image is the \term{set} of all outputs obtained when the \term{function} is evaluated at each \term{element} of the \term{subset}
\item onto
   \subitem A \term{function} that uses all the \term{elements} of the \term{codomain}
\item surjective
   \subitem \term{onto}
\item surjection
   \subitem What a \term{function} is called when it is \term{onto}
\item one-to-one
   \subitem A \term{function} is \term{one-to-one} iff $f(a) = f(b)$ implies that $a = b$ for all $a$ and $b$. It never maps two different \term{elements} to the same place.
\item injective
   \subitem \term{one-to-one}
\item injection
   \subitem What a \term{function} is called when it is \term{one-to-one}
\end{itemize}

\bigskip \indent
\begin{itemize}
\item 1-to-1 correspondence
   \subitem A \term{function} that is both \term{one-to-one} and \term{onto}
\item bijective
   \subitem  What a \term{function} is called when it is a \term{one-to-one correspondence}
\item bijection
   \subitem \term{1-to-1 correspondence}
\item arguments

\item k-ary function

\item arity

\item unary function

\item binary function

\item infix notation

\item prefix notation

\item postfix notation

\item predicate/property

\end{itemize}

\bigskip \indent
\begin{itemize}
\item relation $R$
  \subitem A \term{predicate}, most typically when the domain is a set of \term{$k$-tuples}.
\item k-ary relation
  \subitem A \term{predicate}, whose \term{domain} is a \term{set} of \term{$k$-tuples} $A \times \cdot \cdot \cdot \times A$.
\item k-ary relation on A
  \subitem A \term{predicate}, whose \term{domain} is a \term{set} of \term{$k$-tuples} $A \times \cdot \cdot \cdot \times A$.
\item binary relation
  \subitem A \term{2-ary} \term{relation}.
\item equivalence relation
  \subitem A \term{binary relation} that captures the notion of two objects being equal in some feature.
\item reflexive
  \subitem A \term{binary relation} on a \term{set} for which every \term{element} has a \term{relation} to itself.
\item symmetric
  \subitem A \term{binary relation} for which it holds for all $a$ and $b$ in $X$ that if $a$ is related to $b$ then $b$ is related to $a$.
\item transitive
  \subitem A \term{binary relation} that for every $x$, $y$, and $z$, if $xRy$ and $yRz$ implies $xRz$.
\item undirected graph
  \subitem A \term{set} of \term{nodes} with \term{edges} connecting some of the \term{nodes}.
\end{itemize}

\bigskip \indent
\begin{itemize}
\item graph
  \subitem A \term{set} of \term{nodes} with \term{edges} connecting the \term{nodes}.
\item nodes
  \subitem Points in a \term{graph}.
\item vertices
  \subitem Points in a \term{graph}.
\item edges
  \subitem lines in a \term{graph}.
\item degree
  \subitem The number of \term{edges} at a particular \term{node}.
\item labeled graph
  \subitem A \term{graph} with the \term{nodes} and/or \term{edges} labeled.
\item subgraph
  \subitem A \term{graph} $G$ whos \term{nodes} are a \term{subset} of the \term{nodes} in \term{graph} $H$, and the \term{edges} of $G$ are the \term{edges} of $H$ on the corresponding \term{nodes}.
\item path
  \subitem A sequence of \term{nodes} \term{connected} by \term{edges}.
\item simple path
  \subitem A \term{path} that doesn't repeat any \term{nodes}.
\item connected
  \subitem A \term{graph} with a \term{path} between every two \term{nodes} is considered \term{connected}.
\item cycle
  \subitem A \term{path} that starts and ends at the same \term{node}.
\item simple cycle
  \subitem A \term{cycle} that contains at least three nodes and repeats only the first and last \term{nodes}.
\item tree
  \subitem A \term{graph} that is \term{connected} and has no \term{simple cycles}.
\item root
  \subitem A specially designated \term{node} contained in a \term{tree}.
\item leaves
  \subitem A \term{node} of degree 1 in a \term{tree} that is not the \term{root}.
\item directed graph
  \subitem A \term{graph} with all \term{edges} defined as an \term{ordered pair} or all \term{edges} are unidirectional.
\item outdegree
  \subitem The number of directed \term{edges} leaving a \term{node}.
\item indegree
  \subitem The number of directed \term{edges} entering a \term{node}.
\item directed path
  \subitem A \term{path} where all the \term{edges} are \term{directed egdes}.
\item strongly connected
  \subitem A \term{directed graph} is considered \term{strongly connected} if it contains a \term{path} between every two \term{nodes}.
\end{itemize}
   
\bigskip \indent
\begin{itemize}
\item \emph{alphabet} $\Sigma$

\item symbols
   \subitem A member of an alphabet.

\item string over an alphabet
   \subitem A finite \term{sequence} of \term{symbols} from the \term{alphabet}, usually written next to one another and not separated by commas.

\item length 
   \subitem If  $w$ is a \term{string} over  $\Sigma$ , the \term{length} of $w$, written $\abs{w}$, is the number of \term{symbols} that it contains.

\item empty string 
   \subitem The \term{string} of \term{length} zero and is written $\varepsilon$.

\item reverse
   \subitem Written $w^\mathcal{R}$, is the \term{string} obtained by writing $w$ in the opposite order (i.e., $w_{n}w_{n-1}...w_{1}).$

\item substring
   \subitem A \term{subset} of the \term{symbols} in a \term{string}, where the order of the \term{elements} is preserved.
   
\item concatenation
   \subitem An operation that sticks strings from one \term{set} together with \term{strings} from another \term{set}. 

\item lexicographic ordering
   \subitem A natural order structure of the \term{Cartesian product} of two ordered \term{sets}.

\item language
  \subitem A \term{set} of \term{strings}.

\end{itemize}

\bigskip \indent
\begin{itemize}
\item Boolean logic
  \subitem A mathematical system built around the two values TRUE and FALSE.

\item Boolean values
   \subitem The values TRUE or FALSE, often represented by 1 or 0.

\item Boolean operations
  \subitem An operation that follows the rules of \term{Boolean algebra}; each \term{operand} and the result take one of two values

\item negation/NOT 
   \subitem A \term{boolean operation} that produces the complement of the nominated truth value. It's designated with the \term{symbol} $\neg$.
   
\item conjunction/AND
   \subitem  A logical connective that has the value TRUE if both of its \term{operands} are TRUE, otherwise a value of FALSE. It's designated with the \term{symbol}  $\wedge$.

\item disjunction/OR 
   \subitem A  logical connective that has the value TRUE if at least one of its \term{operands} are TRUE, otherwise a value of FALSE. It's designated with the \term{symbol}  $\vee$

\item exclusive or/XOR 
   \subitem A  logical connective that has the value TRUE if either but not both of its two \term{operands} are TRUE. It's designated with the \term{symbol} $\oplus$.

\item equality (operation) 
   \subitem A logical connective that has the value TRUE if both of its \term{operands} have the same value. It's designated with the \term{symbol} $\leftrightarrow$.

\item implication (operation) 
   \subitem Designated by the \term{symbol} $\rightarrow$, it is a logical connective that has the value FALSE if its first \term{operand} is TRUE and its second \term{operand} is FALSE, otherwise it has a value of TRUE.

\item operands
   \subitem A quantity on which an \term{operation} is performed.

\item distributive law for AND and OR
   \subitem A property of binary operations that generalizes the \term{distributive law} from elementary algebra that says, for all $a, b, c$ in $A, (a b) c = (a c) (b c)$. The Boolean version comes in two forms: 
   \subitem $P \vee (Q \wedge R) = (P \vee Q) \wedge (P \vee R)$
   \subitem $p \wedge (Q \vee R) = (P \wedge Q) \vee (P \wedge R)$, and its dual.

\end{itemize}

\end{document}

%%% Local Variables: 
%%% mode: latex
%%% TeX-master: t
%%% compile-command: "latex defineterms.tex && dvipdf defineterms.dvi"
%%% End: 
